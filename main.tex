\documentclass[11pt]{article}
\usepackage{geometry}                % See geometry.pdf to learn the layout options. There are lots.
\geometry{letterpaper}                   % ... or a4paper or a5paper or ... 
\usepackage[parfill]{parskip}    % Activate to begin paragraphs with an empty line rather than an indent
\usepackage[]{graphicx}
\usepackage{amssymb}
\usepackage{epstopdf}
\usepackage{tikz}
\usetikzlibrary{fit,positioning}
\usepackage[position=b]{subcaption}
\usepackage{enumitem}
\usepackage{hyperref}
\usepackage{cleveref}
\usepackage{tabto}
\NumTabs{8}

\usepackage{titlesec}

\titleformat{\section}[runin]
  {\normalfont\Large\bfseries}{\thesection}{0.5em}{}
\titleformat{\subsection}[runin]
  {\normalfont\large\bfseries}{\thesubsection}{0.5em}{}

\title{FY21 LDRD Full Proposal template}
%\author{Jonathan Eisch}
%\date{November 2020}

\begin{document}

%\maketitle
\begin{center}
\normalfont\Large\bfseries Fermilab LDRD Proposal
\end{center}
\section*{Project Title:}

\section*{Principal Investigator:}

\section*{Lead Division/Sector/Section:}

\section*{Co-Investigators (w/institutions):}



\section*{Proposed FY and Total Budgets:} (summary of budget page (in dollars)) 

\begin{table}[h]
\resizebox{\textwidth}{!}{%
\begin{tabular}{|c|c|c|c|c|c|c|}
\hline
            & SWF & SWF OH & M\&S & M\&S OH & Contingency & Total \\ \hline
1/2 yr FY21 &     &        &      &         &             &       \\ \hline
FY22        &     &        &      &         &             &       \\ \hline
FY23        &     &        &      &         &             &       \\ \hline
1/2 yr FY24 &     &        &      &         &             &       \\ \hline
Total       &     &        &      &         &             &       \\ \hline
\end{tabular}%
}
\end{table}

SWF: Salary, Wages, Fringe \tabto{14em} SWF OH: overhead on SWF \\
M\&S: Material and Supplies \tabto{14em} M\&S OH: overhead on M\&S \\
Contingency (estimate of additional funds that might be required with justification)

\section*{Initiative:} 2021 Broad Scope

\section*{Project Description} (150-200 words): Summarize in 150-200 words the scientific/technical objectives of the proposal, methods that will be used, and expected deliverables and their expected impact. This description should be understandable to a technically literate lay reader. 

\newpage
\section*{Significance} (~1-2 pages): Describe the scientific/technical problem that the proposal addresses, explain why this problem is significant, and introduce your novel approach for addressing this problem. Include a critical comparison of your proposed approach with the latest published work and explain how your project would advance the state of the art and influence its field of research. Begin with the “big picture” and funnel the reader to the significance of the specific problem addressed in the proposal. 

\newpage
\section*{Research Plan} (~3-4 pages): Provide a brief overview of your research plan and your specific objectives or aims. For each objective/aim, provide a section with the following:
\begin{enumerate}[label=\alph*]
\item State the objective/aim
\item Describe the scientific hypotheses to be tested or technical concepts to be demonstrated to achieve the objective/aim.
\item Discuss the methods, materials, facilities, protocols to be employed, and techniques for analyzing data and validating results as appropriate.
\item Describe the expected results and impact (e.g., fundamental breakthroughs, enabling technologies)
\item Provide a deliverable(s) for year one (within the first year of funding), year two (if seeking a multi-year project), and at the completion of the project.
\end{enumerate}

\newpage

\section*{Future Funding} (~1/2 page):
\begin{enumerate}[label=\alph*]
\item Describe how your results will be disseminated; include likely journals in which your work would be published and conferences, workshops, and planning activities at which the work would be presented.

\item List the probable future funding sources (sponsors), including DOE programs, other agencies, state agencies, or private sector investment.

\item For each probable funding source:
\begin{enumerate}[label=\roman*]
\item Explain how the sponsor would benefit from the pursuit of this work.
\item Discuss probable contacts with sponsors and plans for responding to current and planned proposal calls.

\item Estimate the likelihood that the sponsor would provide future funding including anticipated range of such funding. 
\end{enumerate}
\end{enumerate}

\section*{References} (not included in 6 page limit): Cite a concise set of relevant literature that supports the scientific/technical significance of your project and the innovativeness of your proposed methodologies.

\section*{Qualifications} (optional, not included in 6 page limit): The C.V. of the PI may be attached and/or a brief statement (~1/2 page) discussing the qualifications of the PI to carry out the proposed research may be described.

\section*{Resource Availability and Recent LDRD Funding} \tabto{0em}
(not included in 6 page limit):
\begin{enumerate}[label=\alph*]
\item Discuss scientific or technical obligations of the investigators that may limit the available time for working on the LDRD project (e.g., other funded research, participation in scientific committees, etc.); use units of FTE’s to estimate the time
\item List other LDRD commitments of the investigators; include both current (funded projects) and pending (new proposal) commitments; use units of FTE’s to estimate the time
\item Summarize accomplishments of funded LDRD projects for the last five years (include project title, investigators, and year of project)
\end{enumerate}

\section*{Budget Table} (not included in 6 page limit, separate document): The last page of the proposal consists of a completed budget table. In the budget table, include a cost breakdown for each objective/ aim discussed in the Research Plan. There is no specific budget limitation, but keep in mind that the LDRD funding has limited resources. If subcontracting work, it should be clear in the proposal what work is being subcontracted and justified why that work is not able to performed within Fermilab.

\end{document}
